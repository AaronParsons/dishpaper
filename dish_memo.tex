\documentclass[11pt]{article}

%aliases
\def\hf{\frac12}

%\usepackage{abbrevs}
\usepackage{natbib}
\usepackage{hyperref}
\usepackage{graphicx}     % could insert ``draft'' between []
\usepackage{caption}
\usepackage{amsmath,amsfonts}
\usepackage{enumitem}
%\usepackage{subcaption}
\pagestyle{empty}

\setlength{\oddsidemargin}{0pt} % there is 1 inch before the
                                % side margin in ``article'' class
\setlength{\textwidth}{6.5in}

\setlength{\voffset}{0pt}
%\setlength{\topmargin}{-36pt}     % there is 1 inch before the
\setlength{\topmargin}{-0.75in}     % there is 1 inch before the
                                % top margin in ``article'' class and
                                % then room for header, etc.
\setlength{\textheight}{9.5in}
%%%%%%%%%%%

\begin{document}
{\centering{\bf\Large HERA : Antenna Design Discussion} \\}
{
\centering{\bf\large Hydrogen Epoch of Reionization Array \\}
}
\vspace*{0.5cm}

\section{Cost Vs. Sensitivity}
    We weigh the cost and sensitivity benefits of the HERA antenna and array. 
In order to accomplish this we define our figure of merit as the inverse of the 
Cost (C) times the sensitivity ($\Delta_{N}^{2}$):
\begin{equation}
\label{eqn:fom}
    F.O.M = [C \times \Delta_{N}^{2}]^{-1}.
\end{equation}

Our sensitivity is proportional to $k^{3}$, $N^{-1}$, $u^{\frac{1}{2}}$, and $\Omega^{\frac{5}{2}}$. $N$ is the 
number of antennas, $u$ is the baseline length, and $\Omega$ is the beam size. Therfore, 
the sensitivity goes as
\begin{equation}
\begin{split}
    \Delta_{N}^{2} &\propto k^{3}N^{-1}u^{\frac{1}{2}} \Omega^{\frac{5}{2}} \\
    \Rightarrow \Delta_{N}^{2} &\propto k^{3}N^{-1} D^{\hf} (D^{-2})^{\frac{5}{4}} \\
    \Rightarrow \Delta_{N}^{2} &\propto k^{3}N^{-1} D^{-2}\\
\end{split}
\end{equation}

In general, the minimum $k$ mode we can work at is 
\begin{equation}
    k_{min} = k_{H} + \Delta{k_{fg}},
\end{equation}
where $k_{H}$ is the $k$-mode given by the geometrical horizon limit of our baseline, and $k_{fg}$ is 
the width of the foregrounds in $k$-space. Assuming that our antennas are placed next to each other, 
\begin{equation}
    k_{H} = \frac{B}{c}\frac{dk}{d\eta},
\end{equation}
where $B$ is the baseline length (in our case $B$ = $D$, the diameter) and $\frac{dk}{d\eta}$ is 
some cosmological constant of proportionality that depends on the redshift.

We will now look at a few limiting cases of $k$ and cost, $C$. In $k$-space our cases are 
\begin{enumerate}
    \item{$k_{H} \gg k_{fg} \Rightarrow \Delta_{N}^{2} \propto D^{1}N^{-1}$}
    \item{$k_{fg} \gg k_{H} \Rightarrow \Delta_{N}^{2} \propto D^{-2}N^{-1}$} 
    \item{$k_{H} \approx k_{fg} \Rightarrow \Delta_{N}^{2} \propto D^{1}N^{-1}$}
    \item{$k > k_{min}$.}
\end{enumerate}

The cost models are defined in terms of $C$. The total cost is proportional to 
\begin{equation}
    C \propto C_{0} + C_{1}N + C_{2}N^{2}
\end{equation}
where $C_{0} \propto D^{0}$, $C_{1} \propto D^{2}$, and $C_{2} \propto D^{0}$, and $N$ is the number of antennas.
The limiting cases for the cost are when
\begin{enumerate}[label=\Alph*.]
    \item{Linearly dominated : $C \approx C_{1}N \approx D^{2}N$}
    \item{Quadratic in $N$ : $C \approx C_{2}N^{2}$}
    \item{Equal contributions : $C_{0} \approx C_{1}N \approx C_{2}N^{2}$}
\end{enumerate}

\begin{table}[htdp]
\caption{This table shows the proportionalities of the FOM as a function of the diameter ($D$) and the 
        number of antennas $N$. Each of the numbers (top row) and letters (left column) signifies the regime
        we are in, as given in the text.}
\begin{center}
\begin{tabular}{|c|c|c|c|c|}
\hline
&         1 &     2 &       3 &         4 \\\hline
A &  $D^{-3}$ &  constant  & $D^{-3}$  & constant \\\hline
B &  $D^{-1}N^{-1}$ & $N^{-1}D^{2}$ & $D^{-1}N^{-1}$ & $N^{-1}D^{2}$\\\hline
C &  $ (DN^{-1} + D^{3} + DN)^{-1}  $  & $(N^{-1}D^{-2} + 1 + N^{-1}D^{2})$ & $(DN^{-1} + D^{3} + DN)^{-1}$ & $(N^{-1}D^{-2} + 1 + N^{-1}D^{2})$ \\
\hline
\end{tabular}
\end{center}
\label{tbl:matrix}
\end{table}

Table \ref{tbl:matrix} gives a break down of every possible scenario. It provides the FOM (in proportionalites)
as a function of the diameter and number of elements. 

Now consider the case where we are in the fixed cost regime so that the figure of merrit is proportional to the 
inverse of the sensitivity. PAPER falls in the case A for costing. That is, we are linearly dominated and our 
cost goes as $C \propto D^{2}N$. Since our cost is constant, we can vary our diameter as $N^{-\hf}$, or vary 
the number of antennas as $D^{-2}$. If we look at the horizon dominated case, then the figure of merit says that 
we should increase the number of antennas to get the highest sensitivit for our fixed cost. This is because, 
$D \propto N^{-\hf} \Rightarrow N^{\frac{3}{2}}$, therefore increasing the number of antennas will maximize the 
FOM. 
However, if we are forground dominated then the FOM is constant and we want to twiddle the knobs so that we have
the $D \propto N^{-hf}$. Further discussion of this scenario is required. Note, that the scenarios repeat with 
cases 3 and 4.











\end{document}

\documentclass[preprint]{aastex}  % USE THIS TO MAKE BIB, THEN FORMAT USING EMULATEAPJ
%\documentclass[twocolumn,numberedappendix]{emulateapj}
\shorttitle{Transit Dish for 21cm Intensity Mapping}
\shortauthors{XXX Authors}

\usepackage{amsmath}
\usepackage{graphicx}
\usepackage[figuresright]{rotating}
%\usepackage{rotating}
\usepackage{natbib}
%\usepackage{pdflscape}
%\usepackage{lscape}
\citestyle{aa}

\newcommand{\BigO}[1]{\mathcal{O}(#1)}
\def\k{\mathbf{k}}
\def\r{\mathbf{r}}
\def\V{\mathbb{V}}
\def\At{\tilde{A}}
\def\Vt{\tilde{V}}
\def\Tt{\tilde{T}}
\def\tb{\langle T_b\rangle}

\begin{document}
\title{A Transit Dish Design for High-Redshift 21cm Intensity Mapping Experiments}

\author{
%Aaron R. Parsons\altaffilmark{1,2},
%Adrian Liu\altaffilmark{1},
%James E. Aguirre\altaffilmark{3},
%Zaki S. Ali\altaffilmark{1},
%Richard F. Bradley\altaffilmark{4,5,6},
%Chris L.  Carilli\altaffilmark{7},
%David R. DeBoer\altaffilmark{2},
%Daniel C. Jacobs\altaffilmark{8},
%David F. Moore\altaffilmark{3},
%Jonathan C. Pober\altaffilmark{1},
}

%\altaffiltext{1}{Astronomy Dept., U. California, Berkeley, CA}
%\altaffiltext{2}{Radio Astronomy Lab., U. California, Berkeley, CA}
%\altaffiltext{3}{Dept. of Physics and Astronomy, U. Pennsylvania, Philadelphia, PA}
%\altaffiltext{4}{Dept. of Electrical and Computer Engineering, U. Virginia, Charlottesville, VA}
%\altaffiltext{5}{National Radio Astronomy Obs., Charlottesville, VA}
%\altaffiltext{6}{Dept. of Astronomy, U. Virginia, Charlottesville, VA}
%\altaffiltext{7}{National Radio Astronomy Obs., Socorro, NM}
%\altaffiltext{8}{School of Earth and Space Exploration, Arizona State U., Tempe, AZ}
%\altaffiltext{9}{Square Kilometer Array, South Africa Project, Cape Town, South Africa}
%\altaffiltext{10}{Cavendish Lab., Cambridge, UK}

\begin{abstract}
\end{abstract}

\section{Introduction}

\begin{itemize}
\item importance of frequency smoothness
\item collecting area
\end{itemize}

\section{Background}
\label{sec:background}

\begin{itemize}
\item $\tau$-modes, wedge, EoR window
\item geometric interpretation of $\tau$-modes
\end{itemize}

\section{Geometric Constraints}
\label{sec:geometry}

\begin{itemize}
\item first principles of EoR window, how that maps to a specification for design
\item cost analysis
\item critically constrained design
\item symmetric on-axis parabaloid, for reasons of symmetry in polarization response
\end{itemize}

\section{Design and Construction}
\label{sec:design}

\begin{itemize}
\item bent pipes as approximation to parabola
\item faceting (rms deviation from parabaloid)
\item shielding (screens)
\item splash cone
\item hub
\item feed suspension
\item materials, PVC selection
\item design lifetime (wood, PVC)
\end{itemize}

\section{Simulated Performance}
\label{sec:sim}

\begin{itemize}
\item beam pattern
\item delay performance (E-M modeling)
\end{itemize}

\section{Fabrication and Deployment}
\label{sec:deploy}

\begin{itemize}
\item lessons/principles
\item process to precision to specifications on precision
\end{itemize}

\section{Reflectometry Test Setup}
\label{sec:reflect}

\begin{itemize}
\item reflectometry
\item hardware
\item calibration of the good VNA
\item feed height test
\item configurations
\item window function used in delay transform
\end{itemize}

\section{Results}
\label{sec:results}

\begin{itemize}
\item delay spectrum for configurations
\item cost
\item photos of constructed element
\item XXX hook up receiver and to a sky test?
\item measured parabolicity
\item why were we right in the antenuation per reflection?
\item does the cone help?
\item how well did we place everything?
\item ways to ensure spec in field
\item mention extender as unnecessary in flat deployments
\end{itemize}

\section{Conclusion}
\label{sec:conclusion}

\begin{itemize}
\item relevance to HERA, project cost
\item link Pober et al. (2014) sensitivity/science
\item polarization matching
\item frequency coverage, need for a feed re-design.
\end{itemize}

\section{Acknowledgment}

We would like to thank SKA-SA for the site infrastructure, maintenance, and observing support
that has made this work possible, as well as the significant efforts of the staff at
NRAO's Green Bank and Charlottesville sites.  AP would like to thank M. McQuinn and A. Lidz
for helpful discussions and ionization models.
The PAPER project is supported
by the National Science Foundation (awards 0804508,
1129258, and 1125558), and a generous grant
from the Mt. Cuba Astronomical Association.

% ---------------------------------------------------------------------
% ---------------------------------------------------------------------
% ---------------------------------------------------------------------

\bibliographystyle{apj}
\bibliography{biblio}

\end{document}


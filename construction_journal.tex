\documentclass[12pt, letter]{article}
\special{8.5in , 11in}
\usepackage[margin = 1in, letterpaper]{geometry}
\usepackage{multirow}
\usepackage{hyperref}

\begin{document}
\title{Construction Journal}
This pdf includes some note on what we have been building, what are the things that came out in reality (different / as expected?).

\section{Notes on Vol. I}
Construction of the dish began in the courtyard of HFA in Berkeley, CA.
Some lessons that were learned from that construction I call Vol. I :
\begin{enumerate}
    \item{The size of the sono-tube is set by the distance between the center of
the walls. In order to get the spacing correct of the $16$ PVC pipes around the
tube, make sure to measure the outer diameter and divide $16$ for the arc. Since both inner
and outer tubes were drilled with 16 holes the same way or measurement
method before realizing the problem, the tubes fit in Vol. I.}
    \item{Use less water in the concrete!}
    \item{Special care of nails on fixing PVC sleeves while pouring concrete.}
    \item{14 meters of PVC is too droopy. Creating a perfect (or near perfect)
may be hard by just anchoring the two ends. Support is needed. See Dave's new
design (6/7/2013).}
    \item{3 Segments of all sizes of PVCs we have in hand were
      connecting using 2 couplers sitting in HFA}
\end{enumerate}

\section{Notes on Vol. II}
We are currently building the first actual thing in Carl's backyard, the first challenge with that is the slanted ground. So a lot of the distances, angles need to make use of the theodolite and some tools to be as accurate as possible.
 
As we started building, there are few things to note, such as distances, heights, angles, errors due to human mistakes... Ways to solve them..

\begin{enumberate}
    \item{12 spokes + 12 supporting spars, not 16 as we previously designed with the 1st hub at HFA (vol. I)}
    \item{sleeves length is 15'' x 24 out of 2.5'' PVC}
    \item{Hub launch angle is 2.86 $^{\circ}$ using 18'' and 36'', see Equation \ref{launchangle}}
    \item{
      \begin{table}[!h]
        \centering
        \begin{tabular}{|c|c|} \hline
          
          Sono Tube & Measured Outer Circumference \\ \hline
          36'' & 114$\frac{7}{8}$''  \\ \hline
          20'' & 64$\frac{2}{16}$'' \\ \hline
          18'' & 58$\frac{3}{16}$'' \\ \hline
        \end{tabular}
        \caption{The measured circumference (outer) of the sono tubes as we learned from Vol. I, it is bigger than simply $R_{listed} \times 2\pi$ \label{Tab:circum_sono}.}
      \end{table}

    \item{gave up the original supporting telephone pole design, now employed a
    equilateral traingle design; pole to pole distance : $14.6$m = $574.8$'' = $47.9$',
    pole to center distance: $8.43$m = $331.9$'' =  $27.66$'}
    \item{made 12 spars of length $288.625$'', correct length should be $291.3$'',
    lost due to neglected built in coupler overlapping region}
    \item{made 12 spokes of $218.5$'', but half of them are plain on both
      ends, half of them have bell coupler at one end}
\end{enumerate}

\begin{equation}
\theta = \arctan{\frac{r}{2f}} 
\label{launchangle}
\end{equation}
where $r = 18 \pm 0.29$'' since circumference of the $36$'' hub is greater than $36 * 2 *\pi$(overall hub radius) and $f = 4.5$m


\subsection{Pole measurements \ surveying}
We surveyed $3$ poles plus the center based on the 1st pivoted pole, we measured the angles from center to each pair of poles as listed in Table \ref{Tab:az_at_poles}. Before the 2 poles were drilld, we measured the difference in height for pairs of poles and with the center using a theodolite, plywoods, and tape measure. These numbers came out to be very encouraging, the angles were close to $120^\circ$ and the distance from each point to the other at the same horizonal level is approximately what we wanted, see Table \ref{Tab:survey_dist_b4_2_holes} for how close is close.\\

After the two holes for the poles were drilled, in order for them to be easier to fit in the hole, they were drilled $16$'' in diameter and about $37$'' in depth. There were wider than the hole for the first pole, hence harder to keep it straight up, we will use more manpower and plywood to place it in position. One problem that arises with the bigger hole: the pole is now tilted to one side, as a result the center of the pole is not concentric on the surveryed center. We can try to shift the pole to match the surveyed center or we can work around with the error resulting from that.\\

A method that Zaki and I was using when we pivoted the first pole (these poles do not have same radius at all heights):
\begin{enumerate}
\item{drill 2 screws on the pole separated by some ``far'' distances; these screws are centered independently by eyeballing only the width of the pole at that height}
\item{Check for centering of screws \bf{Individually} from far away }
\item{fix a string connecting two screws}
\item{setup the theodolite pointing to the string and move the viewer vertically to check if string lines up with black vertical crosshair line}
\item{Jiggle the pole around and make the last step work}

To deal with the slanted center for the hub, that part of the ground is craved / flattened. As a result, the center is now only an estimate to the previous surveyed rebarred point. What we can do perhaps is to figure the center after pivoting all 3 poles since the holes are drilled, there isn't a lot of room for adjustments at the poles. \\


The survey points in the table is defined as:
Pole 1 = the very first pivot pole, hand drilled hole
Pole 2 = The highest pole near the fence
Pole 3 = the pole closes to the shed, near where our cars park \\ 
 

\begin{table}[!h]
\centering
\begin{tabular}{|c|c|} \hline

Point & Angle  \\ \hline
Pole 1 & 48$^\circ$20'0'' \\ \hline
Pole 2 & 286$^\circ$21'41.5'' \\ \hline
Pole 3 & 168$^\circ$21'0'' \\ \hline


\end{tabular}
\caption{The table lists the azimuthal angle at each pole from true North. At this point, pole 2 and pole 3 were drilled, so we estimated where the rebars were surveyed previously.  \label{Tab:az_at_poles}.}
\end{table}



\begin{table}[!h]
\centering
\begin{tabular}{|c|c|c|} \hline

Point 1 & Point 2 & \Delta angle \\ \hline
Pole 1 & Pole 2 & 121^$\circ$58'18.5'' \\ \hline
Pole 2 & Pole 3 & 118$^\circ$41.5'' \\ \hlane
Pole 1 & Pole 3 &  120^$\circ$1'\\ \hline

\end{tabular}
\caption{The difference between the azimuthal angles at each pole. \label{Tab:diff_az_poles}}
\end{table}



\begin{table}[!h]
\centering
\begin{tabular}{|c|c|c|} \hline

Point 1 & Point 2 & Distance \\ \hline
Pole 1 & Pole 2 & 48'\frac{1}{4}'' \\ \hline
Pole 1 & Center & 332'' \\ \hlne
Pole 1 & Pole 3 & 47'\frac{2}{3}'' \\ \hline

\end{tabular}
\caption{The table lists the distances between poles and center before drilling the 2 holes after locating the point of same horion using a theodolite and plywood\label{Tab:survey_dist_b4_2holes}.}
\end{table}


‹\begin{table}[!h]
\centering
\begin{tabular}{|c|c|c|} \hline

Point 1 & Point 2 & Height \\ \hline
Center & Pole 1 & 22\frac{3}{4}'' \\ \hline
Center & Pole 2 & 4\frac{1}{4}'' \\ \hline
Center & Pole 3 & 31'' \\ \hline


\end{tabular}
\caption{The table lists the height between poles and center before drilling the 2 holes, i.e. how slanted from pole to pole\label{Tab:h_diff_pole}.}
\end{table}

\end{document}
